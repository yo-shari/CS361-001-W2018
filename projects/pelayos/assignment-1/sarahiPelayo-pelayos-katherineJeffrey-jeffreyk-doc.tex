\documentclass[12pt]{article}
\title{Vision Statement: Well Liv}
\author{Sarahi Pelayo \and Katherine Jeffrey}
\begin{document}
\maketitle
\section{The Problem}

Many people experience mental and physical trauma from Sexual Assault, PTSD, Abuse, and Depression, but they might not know how to talk about it. If they don�t have people in their lives they can talk to or access to professional help they might not know about other helpful options. It�s important for victims of these problems to understand what�s happened to them so they can find ways to effectively deal with them.

It would also be good for friends and relatives of victims to know how to help these victims or at least how to connect them to useful resources. Spreading awareness is the first step to helping people face their problems. Many of these problems can result in suicide attempts by the victims, which can be avoided by getting them help. 

\section{Evidence}

Based on a study published on 2007 about 19\% of women and 5-6\% of men will be sexually assaulted during their time at college \cite{csas}. And of those only 12\% report the assault to the police \cite{csas}.  This application will help reduces reasons of not reporting like not knowing where to start or fear of being treated badly, or not being sure if what they experienced counts as an assault.

PTSD United reports 71\% of women in the military develop PTSD due to sexual assault while serving \cite{ptsd}. However, not just military personnel can have PTSD. It develops after a serious trauma such as living through a natural disaster or sudden emotional losses. In the US 70\% of people experience a traumatic event and 20\% of them develop PTSD \cite{ptsd}. Many people with this mental disorder might not even know they have it or how to handle the symptoms and they need to be connected with professional help. 

Depression is a serious disorder that is difficult for many people to talk about, even though it�s common. According to the Anxiety and Depression Association of America, �In 2015, around 16.1 million adults aged 18 years or older in the U.S. had experienced at least one major depressive episode in the last year, which represented 6.7 percent of all American adults.� \cite{nimh}. 

It can be extremely difficult for many victims to report their abuse or even talk about it with their friends and family. It is especially difficult for children and adolescents who might not fully understand what happened to them or still fear their abuser \cite{cccp}. About 60\% of abusers are people the victims know such as neighbors and friends \cite{usdj}. �Only 16\% of all rapes were reported to law enforcement� \cite{ncjr}. Abuse victims need direct access to education and trustworthy people to talk to, preferably anonymously. 

\section{Examples}

If someone experiences sexual assault at Oregon State University they may speak to an RA to ask for resources, but the RA is required to report it something the victim may not want. The app gives them helpful information, resources on and off campus, and connects them with professional help such as hotlines and websites for support groups. The survivor can then make a more informed decision on whether they would like to report it and what type of help they want.

A survivor of a natural disaster such as an earthquake or a hurricane could develop PTSD and not even realize it. Many people think only soldiers can have PTSD so they might not recognize the symptoms. If they want to get help but don�t know what kind of help they need they can download this app to figure out what their symptoms could mean. They can contact helplines through the app if they want to talk to someone anonymously. 

\section{Software Solution}

To solve these problems we can create a mobile app that shows users facts about these mental disorders and gives them options for connecting with professional help. The front page will tell users to contact 911 in an emergency and also give the numbers of a suicide hotline and poison control. Having these on the first page will quickly get this information to people in immediate danger. 

The first page will also have a spot for the user to enter their pin number to open the app. The pin can be set in the phone�s app settings. Once inside the app the user will have three options: log symptoms, information, and resources. 

Log symptoms will let users select how they�re feeling from lists of symptoms. This will help users process what they�re feeling and allow the app to give them personalized advice. Information will have pages on abuse, sexual assault, PTSD, and depression with lists of common symptoms, potential causes, health risks, and remedies. These pages will also include contact info for helplines and support groups. Resources will have pages for helpline numbers and what to say to operators. There will also be pages on shelters for abuse victims, advice for friends and family of victims, and anonymous support groups. 

\section{Features}

Freshman are required to take a online course on sexual assault and consent but after that if students want to access information on that topic they cannot go back to the course. They have to check outside sources. An application would have an information page with on statistics about sexual assault, depression and PTSD as well as definitions. 

With the symptom log  can better identify what category of help they might need. The symptom log with be integrated with a calendar so the user can track patterns over time. There would be a resources page with information to: 
\begin{itemize}
    \item Local Hospitals
    \item Local Centers Against Rape and Domestic Violence 
    \item Non-Emergency Police Number
    \item Counseling and Psychological Services
    \item Sexual Assault Support Services
    \item Suicide Hotline
    \item Poison Control Hotline
    \item ect.
\end{itemize}

Another feature might be for those who do not want to report but want to practice self help. It could give information on free meditation classes, free exercise classes, peer support groups, local shelters where time can be spent with animals, and other self help tactics.

\section{Risks}

A challenge will be making the app visible to survivors without it being obvious to attackers. The application needs to be marketed so that people who need it are likely to download it without it being so recognizable to attacker. To combat intrusion we are implementing a passcode to unlock the application.  Because the application features a section on how to help a friend dealing with trauma to an outsider the purpose a person with our application is not immediately clear. Privacy can be maintained through this and well as the ability technology has provided most devices to be able to quickly delete the app where it may be redownloaded again in the future. Marketing for the application that will be widely viewed by the public may highlight the app as a good source for self care. It can also be pitched as an application to get information to help others. 

Users can have application on their phone without fear that others having seen advertisement might make the assumption that they are dealing with depression or trauma from sexual abuse. The passcode is one of this apps defining features. The idea of locking access to an application is not new however it hasn't been used in applications of this type. Others have instead disguised the app name and icon to appear to be another type of application for example. 

Another challenge is applications true usefulness. To build something that will really help focus during development of the software must be on battery consumption and specifically to keep it to a minimum. In an emergency the application won�t be accessible unless the device has power.

\section{Resources}

Some of the things we will include information to national hotlines and help centers. In addition regional and local information on women�s centers, shelters, and professional psychological help. The resources in the self care section will need also need to be tailored to what is available locally. On the development side some more resources include a database to store information for the application and the users information in relation to their symptom log. Then the mobile application itself. The software used for it will need to address multiple operating systems.

\bibliography{assignment-1-bib}
\bibliographystyle{plain}

\end{document}
